\documentclass[usenames, dvipsnames]{beamer}
%% Possible paper sizes: a0, a0b, a1, a2, a3, a4.
%% Possible orientations: portrait, landscape
%% Font sizes can be changed using the scale option.
\usepackage[size=a0,orientation=portrait,scale=1.6]{beamerposter}
\usepackage{xcolor}
\usetheme{LLT-poster}
% \usecolortheme{ComingClean}
\usecolortheme{Moris}
% \usecolortheme{ConspiciousCreep}  %% VERY garish.

\usepackage[utf8]{inputenc}
\usepackage[T1]{fontenc}
\usepackage{graphicx}
% \usepackage{libertine}
% \usepackage{tgbonum}
% \usepackage{tgpagella}
\usepackage{lmodern}
% \usepackage{times}
% \usepackage{utopia}
% \usepackage{palatino}
% \usepackage[scaled=0.92]{inconsolata}
% \usepackage[libertine]{newtxmath}
% \renewcommand{\familydefault}{qbk}
% \renewcommand{\familydefault}{qpl}
% \renewcommand{\familydefault}{lmr}
% \renewcommand{\familydefault}{ptm}
% \renewcommand{\familydefault}{put}
% \renewcommand{\familydefault}{ppl}
% \renewcommand{\familydefault}{lmtt}
% \renewcommand{\familydefault}{lmss}
% \usepackage[scaled=0.8]{helvet}
% \usepackage[helvet]{sfmath}
% \everymath={\sf}

\usepackage{amsmath}
\usepackage{amssymb}
% \usepackage{mwe}
% \usepackage{enumitem}
% \setlist[itemize]{leftmargin=*}

% Neccessary since lmodern fonts declare only small math symbols
\DeclareFontShape{OMX}{cmex}{m}{n}{
  <-7.5> cmex7
  <7.5-8.5> cmex8
  <8.5-9.5> cmex9
  <9.5-> cmex10
}{}
\SetSymbolFont{largesymbols}{normal}{OMX}{cmex}{m}{n}
\SetSymbolFont{largesymbols}{bold}  {OMX}{cmex}{m}{n}
\newcommand{\Losc}{\ensuremath{L^{\text{osc}}}}
\newcommand{\Lcoh}{\ensuremath{L^{\text{coh}}}}
\newcommand{\Lc}{\ensuremath{L^{\text{coh}}}}
\newcommand{\Dm}{\ensuremath{\Delta m^2}}
\newcommand{\Important}{\textcolor{Bittersweet}}
\newcommand{\Regular}{\textcolor{AccentBlue}}
\newcommand{\regitem}{\item[\Regular{$\bullet$}]}
\newcommand{\impitem}{\item[\Important{$\bullet$}]}





\author[]{\textbf{Konstantin Treskov} on behalf of the Daya Bay collaboration}
\title{Experimental study of decoherence effects\\ in neutrino oscillations in
Daya Bay}
\institute{Joint Institute for Nuclear Research, Dubna, Russia}

\begin{document}
\begin{frame}[fragile]
\begin{columns}[T]

%%%% First Column
\begin{column}{.5\textwidth}

\begin{block}{Conventional neutrino oscillations}
\begin{itemize}
    \regitem Convetional approach to neutrino oscillations relies on a number of
        assumptions:
        \begin{itemize}
            \item Flavor state is a superposition of mass states with
                identical energies:
                \begin{equation*}
                    | \nu_\alpha \rangle = \sum_{i=1}^{3} V^*_{\alpha i}\, |
                    \nu_i(p) \rangle
                \end{equation*}
            \item Production and detection occur coherently.
            \item Mass states travel at speed of light.
        \end{itemize}
    \item That leads to extensively experimentally studied oscillation probability
    \begin{equation*}
        P_{\alpha\beta}(L) = \sum_{i,k=1}^3 V^*_{\alpha i} V_{\beta i}
        V^*_{\beta k}
        V_{\alpha k} \exp \left( -i \frac{L}{\Losc_{ik}}\right), \quad
        \Important{\ensuremath{\Losc_{ik} = \dfrac{4 \pi E}{\Dm_{ik}}}}
    \end{equation*}
    \impitem There are internal inconsistencies within that approach:
        \begin{itemize}
            \item Identical energies are not possible due to decay kinematics.
            \item The definite energy means spacial delocalization -- what is the L?
            \item The coherence of production and detection should be proven.
            \item Equal energy assumption is not Lorentz-invariant.
        \end{itemize}
\end{itemize}
\end{block}

\begin{block}{Neutrino oscillation with wave packets}
  \begin{itemize}
      \item The issues above can be addressed when one consider neutrino
          flavor state as coherent superposition of mass states with different
          momenta:
        \begin{equation*}
\quad\, P_{\alpha\beta}(L)  =\sum_{k,\,j=1}^3\frac{ V^{\phantom\dagger}_{k \beta }V^*_{\alpha k}V^{\phantom\dagger}_{j \alpha }  V^*_{\beta j} }{\sqrt[4]{1 +
    \left(L/L^{\text{d}}_{kj}\right)^2}}\,\,
    \exp{\left[- \frac{\left(L/L^\text{coh}_{kj}\right)^2}{1+\left(L/L^{\text{d}}_{kj}\right)^2} -\mathrm{D}^2_{kj}\right]}
    \text{e}^{-i(\varphi_{kj} + \varphi^d_{kj})},
\label{eq:ossc}
\end{equation*}
  \end{itemize}
\end{block}


\end{column}

%%%% Second Column
\begin{column}{.5\textwidth}
\end{column}
\end{columns}

\end{frame}
\end{document}
